%Dokumentklasse
\documentclass[bibtotoc, a4paper,12pt]{scrreprt}

%\usepackage[onehalfspacing]{setspace}
% ============= Packages =============

% Schriftart
\usepackage{helvet}
\renewcommand{\familydefault}{\sfdefault}

% Blattränder
\usepackage[left = 4cm, right = 1.5cm, top = 3.0cm, bottom = 2.0cm, footskip = 1cm]{geometry}

% Kommentieren von Blöcken\\
\usepackage{comment}

% Dokumentinformationen
\usepackage[
	pdftitle={Globale Suche einer verteilten Cloudservicearchitektur},
	pdfsubject={},
	pdfauthor={David Stahl},
	pdfkeywords={},	
	%Links nicht einrahmen
	hidelinks
]{hyperref}

% Standard Packages
\usepackage[utf8]{inputenc}
\usepackage[ngerman]{babel}
\usepackage[T1]{fontenc}
\usepackage{graphicx, subfig}
\graphicspath{{img/}}
\usepackage{fancyhdr}
\usepackage{lmodern}

% Colors
\usepackage{color}
\usepackage[usenames,dvipsnames,svgnames,table]{xcolor}

% Listening für Source Code 
\usepackage{listings}
\renewcommand{\lstlistingname}{Quelltext}
\renewcommand{\lstlistlistingname}{Quelltext und Daten}

% For XML
\lstdefinelanguage{xml}{
    basicstyle=\normalfont\ttfamily,
    commentstyle=\color{eclipseStrings}, % style of comment
    stringstyle=\color{eclipseKeywords}, % style of strings
    numbers=left,
    numberstyle=\scriptsize,
    stepnumber=1,
    numbersep=8pt,
    showstringspaces=false,
    breaklines=true,
    frame=lines,
    string=[s]{"}{"},
    comment=[l]{:\ "},
    morecomment=[l]{:"},
    basicstyle=\footnotesize,
    emph={edmx:Edmx, Property, edmx:DataServices, Schema, EntityType, Key, PropertyRef, EntityContainer, EntitySet, atom:link},emphstyle={\color{NavyBlue}}
}

% For JSON
\definecolor{eclipseStrings}{RGB}{42,0.0,255}
\definecolor{eclipseKeywords}{RGB}{127,0,85}
\colorlet{numb}{magenta!60!black}

\lstdefinelanguage{json}{
    basicstyle=\normalfont\ttfamily,
    commentstyle=\color{eclipseStrings}, % style of comment
    stringstyle=\color{eclipseKeywords}, % style of strings
    numbers=left,
    numberstyle=\scriptsize,
    stepnumber=1,
    numbersep=8pt,
    showstringspaces=false,
    breaklines=true,
    frame=lines,
    string=[s]{"}{"},
    comment=[l]{:\ "},
    morecomment=[l]{:"},
    literate=
        *{0}{{{\color{numb}0}}}{1}
         {1}{{{\color{numb}1}}}{1}
         {2}{{{\color{numb}2}}}{1}
         {3}{{{\color{numb}3}}}{1}
         {4}{{{\color{numb}4}}}{1}
         {5}{{{\color{numb}5}}}{1}
         {6}{{{\color{numb}6}}}{1}
         {7}{{{\color{numb}7}}}{1}
         {8}{{{\color{numb}8}}}{1}
         {9}{{{\color{numb}9}}}{1}
}

% For SQL
\lstdefinelanguage{SQL}{
    basicstyle=\normalfont\ttfamily,
    commentstyle=\color{eclipseStrings}, % style of comment
    stringstyle=\color{eclipseKeywords}, % style of strings
    numbers=left,
    numberstyle=\scriptsize,
    stepnumber=1,
    numbersep=8pt,
    showstringspaces=false,
    breaklines=true,
    frame=lines,
    string=[s]{"}{"},
    comment=[l]{:\ "},
    morecomment=[l]{:"},
    basicstyle=\footnotesize,
    emph={SELECT, FROM, AS, INNER, JOIN, WHERE, ON, ORDER, BY, INTO, CORRESPONDING, FIELDS, OF, TABLE, UP, TO, OFFSET, ROWS, ASCENDING, DESCENDING, AND, OR},emphstyle={\color{NavyBlue}}
}

% For YAML
\newcommand\YAMLcolonstyle{\color{red}\mdseries}
\newcommand\YAMLkeystyle{\color{black}\bfseries}
\newcommand\YAMLvaluestyle{\color{blue}\mdseries}

\makeatletter

% here is a macro expanding to the name of the language
% (handy if you decide to change it further down the road)
\newcommand\language@yaml{yaml}

\expandafter\expandafter\expandafter\lstdefinelanguage
\expandafter{\language@yaml}
{
basicstyle=\normalfont\ttfamily,
    commentstyle=\color{eclipseStrings}, % style of comment
    stringstyle=\color{eclipseKeywords}, % style of strings
    numbers=left,
    numberstyle=\scriptsize,
    stepnumber=1,
    numbersep=8pt,
    showstringspaces=false,
    breaklines=true,
    frame=lines,
  comment=[l]{\#},
  morecomment=[s]{/*}{*/},
  commentstyle=\color{purple}\ttfamily,
  stringstyle=\YAMLvaluestyle\ttfamily,
  moredelim=[l][\color{orange}]{\&},
  moredelim=[l][\color{magenta}]{*},
  moredelim=**[il][\YAMLcolonstyle{:}\YAMLvaluestyle]{:},   % switch to value style at :
  morestring=[b]',
  morestring=[b]",
  literate =    {---}{{\ProcessThreeDashes}}3
                {>}{{\textcolor{red}\textgreater}}1     
                {|}{{\textcolor{red}\textbar}}1 
                {\ -\ }{{\mdseries\ -\ }}3,
}

% switch to key style at EOL
\lst@AddToHook{EveryLine}{\ifx\lst@language\language@yaml\YAMLkeystyle\fi}
\makeatother

\newcommand\ProcessThreeDashes{\llap{\color{cyan}\mdseries-{-}-}}



%Package für Subpackages
\usepackage{float}

%Package für Abkürzungsverzeichnis
\usepackage[printonlyused]{acronym}

% Tikzplot package
\usepackage{tikz}

% PDF Package
\usepackage{pdfpages}

% Matplotlib Plots
\usepackage{pgfplots}
\pgfplotsset{compat=newest}
\usepgfplotslibrary{groupplots}
\usepgfplotslibrary{dateplot}

% Zeilenabstand
\usepackage[onehalfspacing]{setspace}

% zusätzliche Schriftzeichen der American Mathematical Society
\usepackage{amsfonts}
\usepackage{amsmath}

%nicht einrücken nach Absatz
%\setlength{\parindent}{0pt}


% ============= Kopf- und Fußzeile =============
\pagestyle{fancy}
%
\lhead{}
\chead{}
\rhead{\slshape \leftmark}
%%
\lfoot{}
\cfoot{\thepage}
\rfoot{}
%%
\renewcommand{\headrulewidth}{0.4pt}
\renewcommand{\footrulewidth}{0pt}

% ============= Package Einstellungen & Sonstiges ============= 
%Besondere Trennungen
\hyphenation{De-zi-mal-tren-nung}


% ============= Dokumentbeginn =============

\begin{document}

%Seiten ohne Kopf- und Fußzeile sowie Seitenzahl
\pagestyle{empty}

\begin{center}
\begin{tabular}{p{\textwidth}}



\includegraphics[scale=0.4]{img/thm.jpg}
\hspace{\fill}
\includegraphics[scale=0.035]{img/studiumplus.png}


\\

\begin{center}
\large{Projektstudiumbericht \\}
\end{center}

\\

\begin{center}
Thema:
\end{center}

\\

\begin{center}
\textbf{\Large{Entwicklung eines Test-Frameworks}}
\end{center}

\\
\\
\\
\\
\\
\\
\\

\begin{center}

\begin{tabular}{lll}
\textbf{Eingereicht bei:} & & Prof. Dr. Peter Hohmann \\
\textbf{Vorgelegt von:} & &	David Stahl \\
& & Bornbachstraße 2a \\
& & 35647 Waldsolms \\
\textbf{Matrikelnummer:} & & 5238275 \\
\textbf{Unternehmen:} & & Rittal GmbH \& Co. KG, Herborn \\
\textbf{Fachbetreuer:} & & Ahmad Abrass \\
\textbf{Eingereicht:} & & 13.10.2019 \\
\end{tabular}
\end{center}

\end{tabular}
\end{center}

\pagenumbering{Roman}

\addcontentsline{toc}{chapter}{Sperrvermerk}
\chapter*{Sperrvermerk}

Der Bericht beinhaltet vertrauliche, interne Informationen der Firma Rittal GmbH \& Co. KG. 
Die Weitergabe des Inhaltes der Arbeit und eventuell beiliegender Zeichnungen und Daten im Gesamten oder in Teilen ist grundsätzlich untersagt. Es dürfen keinerlei Kopien oder Abschriften – auch in digitaler Form – gefertigt werden. Ausnahmen bedürfen der schriftlichen Genehmigung der Firma Rittal GmbH \& Co. KG.




% Beendet eine Seite und erzwingt auf den nachfolgenden Seiten die Ausgabe aller Gleitobjekte (z.B. Abbildungen), die bislang definiert, aber noch nicht ausgegeben wurden. Dieser Befehl fügt, falls nötig, eine leere Seite ein, sodaß die nächste Seite nach den Gleitobjekten eine ungerade Seitennummer hat. 
\cleardoubleoddpage

% pagestyle für gesamtes Dokument aktivieren
\pagestyle{fancy}

%Inhaltsverzeichnis
\addcontentsline{toc}{chapter}{Inhaltsverzeichnis}
\tableofcontents

%Abkürzungsverzeichnis
\addcontentsline{toc}{chapter}{Abkürzungsverzeichnis}
\chapter*{Abkürzungsverzeichnis}

\begin{acronym}[SIZEEEEEE]
\acro{DB}{Datenbank}
\acro{DF}{Document Frequency}
\acro{FIFO}{First In - First Out}
\acro{HTML}{Hypertext Markup Language}
\acro{HTTP}{Hypertext Transfer Protocol}
\acro{ID}{Identifikator}
\acro{IDF}{Inverted Document Frequency}
\acro{IR}{Information Retrieval}
\acro{N}{Anzahl der Dokumente}
\acro{NoSQL}{Not only \acs{SQL}}
\acro{REST}{Representational State Transfer}
\acro{SQL}{Structured Query Language}
\acro{TF}{Term Frequency}
\acro{VSM}{Vector Space Model}
\acro{WF}{Weight Frequency}
\acro{WF.IDF}{Weight Frequency - Inverted Document Frequency}
\acro{SIZEEEEEE}{ssssssssssssss}
\end{acronym}


%Verzeichnis aller Bilder
\listoffigures
\addcontentsline{toc}{chapter}{Abbildungsverzeichnis}

%Verzeichnis aller Tabellen
\listoftables
\addcontentsline{toc}{chapter}{Tabellenverzeichnis}

% Verzeichnis aller Listenings
\lstlistoflistings
\addcontentsline{toc}{chapter}{Quelltext und Daten}

\chapter{Einleitung}
\label{sec:einleitung}

\newcounter{mycounter}
\setcounter{mycounter}{\value{page}}
\pagenumbering{arabic}
\setcounter{page}{1}

\section{Problemstellung}

\section{Zielsetzung}

\section{Vorgehensweise}

\chapter{Software-Tests}
\label{sec:softwarequalitaet}

\section{Prüfebenen}

\subsection{Unit-Test}

\subsection{Integrations-Test}

\subsection{System-Test}

\section{Prüfkriterien}

\section{Prüfmethodiken}

\chapter{Verteilte Systeme}
\label{sec:verteiltesysteme}

\section{Systemarchitekturen}

\subsection{Client/Server-Model}

\subsection{Dreistufige Architekturen}

\subsection{Cluster}

\section{Kommunikation}

\subsection{Web Services über HTTP}

\subsection{Message Oriented Middleware mit RabbitMQ}

\chapter{UI Tests für RiPanel Processing}
\label{sec:uitests_rpp}

\section{Anforderungen}

\section{Softwarearchitektur}

\section{Modellierung der Domäne}

\section{Verteilte Systeme}

\subsection{Grafische Benutzeroberfläche}

\subsection{Webservice}

\subsection{Datenbank}

\section{Software-Tests}

\chapter{Schlussfolgerung}

\section{Reflexion}

\section{Aussichten}

%Literaturverzeichnis
\bibliographystyle{unsrtdin}
\bibliography{Literatur}

% Versicherung
\addcontentsline{toc}{chapter}{Versicherung}
\chapter*{Versicherung}

Ich versichere, dass ich diese Arbeit selbstständig verfasst und keine anderen als die angegebenen Hilfsmittel benutzt habe. Unter Literaturverzeichnis habe ich die benutzten Hilfsmitteln kenntlich gemacht. Dies betrifft sowohl wörtlich als auch inhaltlich entnommene Stellen. Die Arbeit hat in dieser Form noch keiner anderen Prüfungsbehörde vorgelegen.  
\\
\\[1.0cm]
\begin{minipage}{0.45\linewidth}
Datum:	10.08.2019 \hspace{35pt} Unterschrift: 
\end{minipage}
\hfill
\begin{minipage}{0.45\linewidth}
\includegraphics[scale=0.4]{img/Unterschrift.jpg}
\end{minipage}

% Anhang
\appendix

\include{10_anhang}

\end{document}